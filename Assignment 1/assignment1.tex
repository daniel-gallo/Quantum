\documentclass{article}

\usepackage{tcolorbox}
\usepackage{amsmath}
\usepackage{amssymb}
\usepackage{braket}
\usepackage{mathtools}

\def\H{\mathcal{H}}
\def\B{\mathcal{B}}
\def\C{\mathbb{C}}
\DeclareMathOperator{\End}{End}
\DeclareMathOperator{\tr}{tr}
\DeclarePairedDelimiterX{\norm}[1]{\lVert}{\rVert}{#1}
\DeclarePairedDelimiterX{\abs}[1]{\lvert}{\rvert}{#1}

\title{Assignment 1}
\author{Daniel Gallo}

\begin{document}
    \maketitle

    \begin{tcolorbox}[title=Exercise 1]
        Consider a Hilbert space $\H$ with basis $\{\ket{1}, \ket{2}\}$ such that $\braket{1|1} = 5$, $\braket{1|2} = -3i$ and $\braket{2|2} = 2$.
        \begin{enumerate}
            \item Let $\ket{v} = (i\sqrt{2}, 3)$ and $\ket{w} = (2\sqrt{2}, 1 + 3i)$. Compute $\norm{\ket{v}}$, $\norm{\ket{w}}$ and $\braket{v | w}$.
            \item Use Gram-Schmidt to obtain an orthonormal basis $\B$.
            \item Find the column vector representation of $(\alpha, \beta)$ with respect to $\B$.
            \item Find row vector representations for $\bra{v}$ and $\bra{w}$ with respect to to the dual basis $\B^*$
            \item Let $\hat{A}$ be a linear operator on $\H$ such that
            \begin{align*}
                \hat{A} \ket{1} &= \ket{1} \\
                \hat{A} \ket{2} &= -\frac{6i}{5} \ket{1} - \ket{2}
            \end{align*}
            Find the matrix representation of $\hat{A}$ with respecto to $\B$.
        \end{enumerate}
    \end{tcolorbox}
    \begin{enumerate}
        \item 
        \begin{equation*}
            \braket{v | v} =
            \begin{bmatrix}
                -i\sqrt{2} & 3
            \end{bmatrix}
            \begin{bmatrix}
                5 & -3i \\
                3i & 2
            \end{bmatrix}
            \begin{bmatrix}
                i\sqrt{2} \\
                3
            \end{bmatrix}
            = 28 - 18\sqrt{2}
        \end{equation*}
        We can conclude that $\norm{\ket{v}} \approx 1.595$
        \begin{equation*}
            \braket{w | w} = 
            \begin{bmatrix}
                2\sqrt{2} & 1 - 3i
            \end{bmatrix}
            \begin{bmatrix}
                5 & -3i \\
                3i & 2
            \end{bmatrix}
            \begin{bmatrix}
                2\sqrt{2} \\
                1 + 3i
            \end{bmatrix}
            = 60 + 36\sqrt{2}
        \end{equation*}
        Thus, $\norm{\ket{w}} \approx 10.531$
        \begin{equation*}
            \braket{v | w} = 
            \begin{bmatrix}
                -i\sqrt{2} & 3
            \end{bmatrix}
            \begin{bmatrix}
                5 & -3i \\
                3i & 2
            \end{bmatrix}
            \begin{bmatrix}
                2\sqrt{2} \\
                1 + 3i
            \end{bmatrix}
            = 6 + 38i
        \end{equation*}

        \item We can set
        \begin{align*}
            \ket{v_1} &= \ket{1} \\
            \ket{v_2} &= \ket{2} - \frac{\braket{1|2}}{\braket{1|1}} \ket{1} = \frac{3i}{5} \ket{1} + \ket{2}
        \end{align*}
        If we normalize them, we get,
        \begin{align*}
            \ket{e_1} &= \frac{1}{\sqrt{5}} \ket{1} \\
            \ket{e_2} &= \frac{3}{\sqrt{5}}i \ket{1} + \sqrt{5}\ket{2}
        \end{align*}

        \item We want to find $x, y \in \C$ such that $x\ket{e_1} + y\ket{e_2} = \alpha\ket{1} + \beta\ket{2}$.
        \begin{equation*}
            \begin{bmatrix}
                x \\
                y
            \end{bmatrix}
            =
            \begin{bmatrix}
                \frac{1}{\sqrt{5}}  & 0 \\
                \frac{3}{\sqrt{5}}i & \sqrt{5}
            \end{bmatrix}^{-1}
            \begin{bmatrix}
                \alpha \\
                \beta
            \end{bmatrix}
            =
            \begin{bmatrix}
                \sqrt{5} \alpha \\
                -\frac{3}{\sqrt{5}}i \alpha + \frac{1}{\sqrt{5}} \beta
            \end{bmatrix}
        \end{equation*}

        \item Let's express $\ket{v}$ and $\ket{w}$ in $\B$.
        \begin{equation*}
            \ket{v} = 
            \begin{bmatrix}
                \sqrt{5} \left(i \sqrt{2}\right) \\
                -\frac{3}{\sqrt{5}}i \left(i \sqrt{2}\right) + \frac{1}{\sqrt{5}} (3)
            \end{bmatrix}_\B
            = 
            \begin{bmatrix}
                \sqrt{10}i \\
                \frac{3}{5} \left(\sqrt{10} + \sqrt{5}\right)
            \end{bmatrix}_\B
        \end{equation*}
        \begin{equation*}
            \ket{v} = 
            \begin{bmatrix}
                \sqrt{5} \left(2\sqrt{2}\right) \\
                -\frac{3}{\sqrt{5}}i \left(2 \sqrt{2}\right) + \frac{1}{\sqrt{5}} (1 + 3i)
            \end{bmatrix}_\B
            = 
            \begin{bmatrix}
                2\sqrt{10} \\
                \frac{\sqrt{5}}{5}\left(1 + \left(3 - 6\sqrt{2}\right)i\right)
            \end{bmatrix}_\B
        \end{equation*}
        Since $\B$ is an orthonormal basis,
        \begin{align*}
            \bra{v} &=
            \begin{bmatrix}
                \sqrt{10}i & \frac{3}{5} \left(\sqrt{10} + \sqrt{5}\right)
            \end{bmatrix} \\
            \bra{w} &=
            \begin{bmatrix}
                2\sqrt{10} & \frac{\sqrt{5}}{5}\left(1 + \left(3 - 6\sqrt{2}\right)i\right)
            \end{bmatrix}
        \end{align*}

        \item
        Let $M$ be the chage of basis matrix (from $\B$ to the original one).
        \begin{equation*}
            M = 
            \begin{bmatrix}
                \frac{1}{\sqrt{5}}  & 0 \\
                \frac{3}{\sqrt{5}}i & \sqrt{5}
            \end{bmatrix}
        \end{equation*}
        Then,
        \begin{align*}
            A &= M^{-1}
            \begin{bmatrix}
                1 & -\frac{6i}{5} \\
                0 & -1
            \end{bmatrix}
            M \\
            &=
            \begin{bmatrix}
                4.6 & -6i \\
                -3.36 & -4.6
            \end{bmatrix}
        \end{align*}
    \end{enumerate}

    \begin{tcolorbox}[title=Exercise 2]
        Let $\H$ be a finite-dimensional Hilbert space.
        \begin{enumerate}
            \item Recall the definition of the Hilbert-Schmidt inner product on $\End(\H)$. Explicitly show that it defines an inner product. 
            \item Find an orthonormal basis on $\End(\H)$ with respect to this inner product. Express your solution in the outer product notation and state the dimension of $\End(\H)$.
            \item Find an orthonormal basis of Hermitian matrices for $\End(\H)$ with respect to this inner product.
        \end{enumerate}
    \end{tcolorbox}

    \begin{enumerate}
        \item The Hilbert-Schmidt inner product is defined as follows
        \begin{align*}
            \langle\cdot,\cdot\rangle \colon \End(\H) &\to \End(\H) \\
            A, B &\mapsto \tr(A^\dagger B)
        \end{align*}
        Let's verify that it is indeed an inner product.
        \begin{enumerate}
            \item \textbf{Linearity} in the second component
            \begin{align*}
                \langle A, \beta_1 B_1 + \beta_2 B_2\rangle &= \tr(A^\dagger (\beta_1 B_1 + \beta_2 B_2)) \\
                &= \tr(\beta_1 A^\dagger B_1 + \beta_2 A^\dagger B_2) \\
                &= \beta_1\tr(A^\dagger B_1) + \beta_2\tr(A^\dagger B_2) \\
                &= \beta_1\langle A, B_1\rangle + \beta_2\langle A, B_2\rangle
            \end{align*}
            \item \textbf{Skew-symmetry}
            \begin{align*}
                \langle A, B\rangle &= \tr(A^\dagger B) = \tr((A^\dagger B)^\dagger)^*\\
                &= \tr(B^\dagger A)^* = \langle B, A\rangle^*
            \end{align*}
            \item \textbf{Positive-definiteness}
            \begin{equation*}
                \langle A, A\rangle = \tr(A^\dagger A) = \sum_{i,j} \abs{A_{ij}}^2 \geq 0
            \end{equation*}
            Furthermore, $\sum_{i,j} \abs{A_{ij}}^2 = 0$ if and only if $A = 0$.
        \end{enumerate}
    \end{enumerate}

\end{document}